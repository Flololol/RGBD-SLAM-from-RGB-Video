\chapter{Related Work}
    With the abundance of mobile recording devices and therefore monocular videos, the estimation of accurate depth maps that can lead to 3D reconstructions is a popular challenge.
    There are many suggested approaches for the estimation of depth maps.
    While \citetitle{roussos2012dense}~\cite{roussos2012dense} and \citetitle{russell2014video}~\cite{russell2014video} focus on the challenge of dynamic objects in a scene, accurate dense 3D reconstructions of stationary scenes is also a difficult challenge.
    This is especially the case for large frame sequences with a lot of camera movement.\\
    In \citetitle{luo2020consistent}~\cite{luo2020consistent} \citeauthor{luo2020consistent} propose a method of estimating depth for monocular videos by adjusting depth estimates of single frames such that temporal and geometric consistency is achieved.
    The initial single frame depth estimations are retrieved through the convolutional network from Google that was trained on the dataset from \citetitle{mannequin}~\cite{mannequin}.
    This network is then fine-tuned on a specific video to satisfy the geometric and temporal constraints.\\
    A method for a live 3D reconstruction with RGB-D input stream is suggested in \citetitle{dai2017bundlefusion}~\cite{dai2017bundlefusion}.
    In this case, the pose estimation is the main challenge because when the problem is scaled up, there is an increasing drift in the pose estimation.
    While other approaches require long offline processing to achieve a globally correct model, their online approach tries to solve this through a hierarchical pose estimation that does not rely on temporal consistency of the frame sequence and contains implicit loop closures.