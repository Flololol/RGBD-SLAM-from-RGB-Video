\chapter{Conclusion}
    Combining the insight we accumulated over the past weeks, we identified two main possibilities for the error in our pipeline, leading to slightly worse results after our pose optimization.
    First, we might have done an amazing job with the minimization of the energy, but the faulty refined depth maps are so far off from the truth, that no optimizer in the world could fit that.
    We would have to investigate more into the origin of the holes, why they occur and how we can make the depth maps more consistent.\\
    The second possiblity is that we did not employ pixel consistency checks like reprojection error and normal deviation to account for occlusion and other sources of error.
    We simply checked if the transformed pixel lands in the target image and realized too late that the optical flow masks computed during the \citetitle{luo2020consistent} step could yield better results for this problem.
    Also our minimization approach may be too unconstrained and has no gradient provided for the optimizer, making it a even more difficult challenge.
    A major part of the bundle fusion paper is their online nonlinear GPU solver written for this exact problem with an intelligent gradient estimation that we simply cannot compete with.
    The reality is probably a mixture of the beforementioned speculations.
    The refined depth maps are definitely quite far from the ground truth as shown in Section~\ref{seq:results} and the energy minimization is a complex problem requiring every optimization possible.\\
    There is one more thing we intended to try.
    In order to confirm the quality of our pose optimization algorithm, we could optimize the initial extrinsics after replacing our depth with the ground truth.
    If properly implemented we should see a lower alignment error when fitting the optimized trajectory to the ground truth.